\documentclass{article}
\usepackage{amsmath}
\usepackage{amssymb}
\usepackage{amsthm}
\usepackage{braket}

\usepackage{graphicx}

\newtheorem{theorem}{Theorem}[section]
\newtheorem{corollary}{Corollary}[theorem]
\newtheorem{lemma}[theorem]{Lemma}

\title{Principles of Quantum Mechanics}
\author{George Lee\\Girton College}
\begin{document}

\maketitle
\section{Introduction}
\subsection{Blah}
sergserhiugyhieru
\section{Dirac Formalism}
State of a quantum system described by a ket member $\ket{\psi}$ of a $\mathbb{C}$-Hilbert space $\mathcal{H}$.  We have the corresponding dual space of bra vectors $\bra{\phi}\in\mathcal{H}^\dagger$.  The space is suffieciently well behaved that for each member $\ket{\psi}$ of the space there is a corresponding dual vector $\bra{\psi}=(\ket{\psi})^*$.  Have sesquilinear inner product $\braket{\sim|\sim}:\mathcal{H}^2\rightarrow\mathbb{C}$ etc.
\\
\\
Physically $\ket{\psi}\mapsto\alpha\ket{\psi}$ transforms the ket to an equivalent state.  For a particle we often choose $\braket{\psi|\psi}=1$, note that we often study systems whioch are idealisations, eg infinite plane waves where $\braket{\psi|\psi}=\infty$.  Even then given $\theta\in\mathbb{R}$ the map $\ket{\psi}\mapsto e^{i\theta}\ket{psi}$ gives equivalent systems, though we find phase differences do matter.
\\
\\
What with this being a vector space we have linear operators on it.  Given $Q:\mathcal{H}\circlearrowright$ linear we can define $Q^\dagger:\mathcal{H}^\dagger\circlearrowright,~Q^\dagger:\bra{\psi}\mapsto(Q\ket{\psi})^\dagger$

\end{document}
