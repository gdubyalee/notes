\documentclass{article}
\usepackage{amsmath}
\usepackage{amssymb}
\usepackage{amsthm}
\usepackage{braket}

\usepackage{graphicx}

\newtheorem{theorem}{Theorem}[section]
\newtheorem{corollary}{Corollary}[theorem]
\newtheorem{lemma}[theorem]{Lemma}

\title{Principles of Quantum Mechanics}
\author{George Lee\\Girton College}
\begin{document}

\maketitle
\section{Introduction}
\subsection{Blah}
sergserhiugyhieru
\section{Dirac Formalism}
\subsection{States and Operators}
State of a quantum system described by a ket member $\ket{\psi}$ of a $\mathbb{C}$-Hilbert space $\mathcal{H}$.  We have the corresponding dual space of bra vectors $\bra{\phi}\in\mathcal{H}^\dagger$.  In terms of notation, $\phi$ is merely a label - in other cases we may label ket vectors by a number or collection of numbers.  The space is suffieciently well behaved that for each member $\ket{\psi}$ of the space there is a corresponding dual vector $\bra{\psi}=(\ket{\psi})^*$.  Have sesquilinear inner product $\braket{\sim|\sim}:\mathcal{H}^2\rightarrow\mathbb{C}$ etc.
\\
\\
Physically $\ket{\psi}\mapsto\alpha\ket{\psi}$ transforms the ket to an equivalent state.  For a particle we often choose $\braket{\psi|\psi}=1$, note that we often study systems whioch are idealisations, eg infinite plane waves where $\braket{\psi|\psi}=\infty$.  Even then given $\theta\in\mathbb{R}$ the map $\ket{\psi}\mapsto e^{i\theta}\ket{psi}$ gives equivalent systems, though we find phase differences do matter.
\\
\\
What with this being a vector space we have linear operators on it.  Given $Q:\mathcal{H}\rightarrow\mathcal{H}$ linear we can define $Q^\dagger:\mathcal{H}^\dagger\rightarrow\mathcal{H}^\dagger$ by $Q^\dagger:\bra{\psi}\mapsto(Q\ket{\psi})^\dagger$.
\\
\\
Eigenstates of $Q$ with eigenvalue $\lambda$ of course satisfy the equation $Q\ket{\psi}=\lambda\ket{\psi}$.  Oh, also $[A,B]=AB-BA$ is the commutator.  Neat identities are associated with it.
\subsection{Observables and measurements}
Operator $Q$ Adjoint/Hermitian if we have $Q^\dagger=Q$ (THIS DOES NOT MAKE SENSE UNLESS WE'RE LIKE, LOL, Say two maps are the same if for all $\ket{\psi}\ket{\phi}$ we have $\braket{\phi |Q|\psi}=\braket{\phi|Q'|\psi}$).  Can show for $Q$ Hermitian:
\begin{itemize}
  \item $Q\ket{\psi}=\lambda\ket{\psi}\implies\lambda\in\mathbb{R}$
  \item $\lambda \neq \lambda',Q\ket{\psi}=\lambda\ket{\psi},Q\ket{\phi}=\lambda'\ket{\phi}\implies \braket{\psi|\phi}=0$
  \item Writing $\mathbb{V}_\lambda$ for the eigenstates of value $\lambda$, we have $\Bar{span}({\mathbb{V}_\lambda:\lambda\in\mathbb{R}})=\mathcal{H}$
\end{itemize}
Maybe I'll write out a bit of a rambling proof later.  Or maybe just check out linear ananlysis or something?
\\
\\
Diagonalising $Q$ is a matter of rewriting the states we care about in terms of $Q$'s eigenstates.  For q $Q$ with a discrete spectrum of eigenvalues with $\forall\lambda: dim(\mathbb{V}_\lambda)$ we can write
$$
  \ket{\psi}=\sum_\lambda\sum_{n=1}^{dim(\mathbb{V}_\lambda)}\ket{n,\lambda}\braket{n,\lambda|\psi}=\sum_\lambda\sum_{n=1}^{dim(\mathbb{V}_\lambda)}\alpha_{n,\lambda}\ket{n,\lambda}
$$
ie,  we have  that the identity operator $Id=\sum_\lambda\sum_{n=1}^{dim(\mathbb{V}_\lambda)}\ket{n,\lambda}\bra{n,\lambda}$.  This simplifies, for a nondegenerate spetrum, ie, when $\forall\lambda:dim(\mathbb{V}_\lambda)=1$ to
$$
  \ket{\psi}=\sum_\lambda\ket{n,\lambda}\braket{\lambda|\psi}=\sum_\lambda\alpha_{\lambda}\ket{\lambda}
$$
Here $\alpha_X=\braket{X|\psi}$.
\\
\\
Physically $Q$ may correspond to a physical quantity, eg when $Q=\hat{\mathcal{H}}$, the Hamiltonian.  When a measurement of the quantity is made we find that $\mathbb{P}(\lambda=\lambda')=\sum_{n=1}^{dim(\mathbb{V}_\lambda')}\braket{n,\lambda'|\psi}^2$.  Upon measurement of value $\lambda$ the wavefunction's non-$\lambda$-eigenvalue components will vanish, and the other components may be renormalised as appropriate.  Thus the expected value of $\lambda$ is
$$
  \mathbb{E}_\psi(Q)=\braket{Q}_\psi=\braket{\psi|Q|\psi}=\sum_\lambda\lambda\mathbb{P}(\lambda)=\sum_\lambda\sum_{n=1}^{dim(\lambda)}\lambda|\alpha_{n,\lambda}|^2
$$
The uncertainty is given by
$$
  (\Delta Q)_\psi^2=\braket{(Q-\braket{Q}_\psi)^2}_\psi=\braket{Q^2}_\psi-\braket{Q}_\psi^2
$$
so that if $Q\ket{\psi}=\lambda\ket{\psi}$ then $\braket{Q}_\psi=\lambda,~(\Delta Q)_\psi=0$.  We can then derive an uncertainty principle, relating the uncertainties of physical quanbtities with their commutators:
\begin{align*}
  \braket{[A,B]}^2&=\braket{[A-\braket{A},B-\braket{B}]}^2
  \\
  &=\braket{(A-\braket{A})(B-\braket{B})-(B-\braket{B})(A-\braket{A})}^2
  \\
  &\leq(|\braket{(A-\braket{A})(B-\braket{B})}|+|\braket{(B-\braket{B})(A-\braket{A})}|)^2
  \\
  &\leq(2(\Delta A)(\Delta B))^2
  \\
  \implies& \frac{1}{2}|\braket{[A,B]}|\leq(\Delta A)(\Delta B)
\end{align*}
where we have used $\braket{AB}\leq\braket{A}\braket{B}$ and also $a,b\in\mathbb{C}\implies[A-a,B-b]=[A,B]$.  A significant example of this is the Heisenburg uncertainty principle: given $[\hat{x},\hat{p}]=i\hbar$ it follows that $\frac{1}{2}\hbar\leq\Delta\hat{x}\Delta\hat{p}$.
\subsection{Time evolution of the Schr\"odinger Equation}
We should hope that a wavefunction should be governed by some appropriate equation.  In particular the assumption often made is that the instantaneous manner in which the system is changing should be describable by some differential equation.  In QM the equation is taken to be
$$
  i\hbar\partial_t\ket{\psi(t)}=\hat{\mathcal{H}}\ket{\psi(t)}
$$
for some linear operator $\hat{\mathcal H}$ so that
$$
  -i\hbar\bra{\psi(t)}=\bra{\psi(t)}\hat{\mathcal{H}^\dagger}
$$
Now given the interpretation of the wavefunction in terms of probability density, we require $1=\braket{\psi|\psi}$ for all time, so that $\forall\ket{\psi(t_0)}$ we find
\begin{align*}
  0&=\partial_t\braket{\psi|\psi}
  \\
  &=\bra{\psi}(\partial_t\ket{\psi})+(\partial_t\bra{\psi})\ket{\psi}
  \\
  &=\frac{1}{i\hbar}\braket{\psi|(\hat{\mathcal{H}}-\hat{\mathcal{H}}^\dagger)|\psi}
\end{align*}
so that $\hat{\mathcal{H}}=\hat{\mathcal{H}}^\dagger$ Hermitian.  $\hat{\mathcal{H}}$ corresponds to the classical notion of energy.  Consider the case where it has a discrete set of eigenvalues $\hat{\mathcal{H}}\ket{n}=E_n\ket{n}$.  Subbing in $\ket{\psi(0)}=\ket{n}$ and writing $\ket\psi=\hat f(t)\ket n$ we find, for $[\hat{\mathcal{H}},t]=0$,
$$
  \hat{\mathcal H}\ket\psi=\hat{\mathcal H}\hat f\ket n=\hat f\hat{\mathcal H}\ket n=\hat fE_n\ket n=E_n\ket\psi,
$$
that is, the state remains stationary - thus $\hat f(t)=e^{i\theta(t)}$ for some $\theta$.  Then we have
$$
  i\hbar\partial_te^{i\theta(t)}\ket n=E_ne^{i\theta(t)}\ket n\implies \ket{\psi(t)}=e^{-\frac{iE_nt}{\hbar}}\ket n
$$
Thus given a general initial state $\ket{\psi(0)}=\sum_n\alpha_n\ket n$ and using linearity of the equation we can write down the solution
$$
  \ket{\psi(t)}=\sum_ne^{-\frac{iE_nt}{\hbar}}\alpha_n\ket n
$$
\subsection{Bases and Representations}
Given an orthonormal ($ON$) representation of states, eg
\begin{align*}
  \ket\psi=\sum_n\alpha_n\ket n
  \\
  \ket\phi=\sum_n\beta_n\ket n
\end{align*}
we find that the inner product is given by $\braket{\phi|\psi}=\sum_n\bar\beta_n\alpha_n$.  If we are given an operator $A$, and we write $A_{mn}=\braket{m|A|n}$ then we have
\begin{align*}
  \ket \phi&=A\ket\psi~(Basis-independent~abstraction)
  \\
  \forall m :\beta_m&=\sum_nA_{mn}\alpha_n~(Basis-dependent)
\end{align*}
Writing $Id=\sum_n\ket n\bra n$ we find $A=Id\circ A\circ Id=\sum_{m,n}\ket mA_{mn}\bra{n}$.  If $Q\ket n =q_n \ket n$ then we have $Q=\sum_n \ket n q_n \bra n$.  In some circumstances it is then appropriate to extend a $\mathbb C $-valued function $f$ to $Q$ via $f(Q)=\sum_n \ket n f(q_n) \bra n$.  Note then eg, for $f(x)=\frac{1}{x}$ we have $Q\circ f(Q)=\sum_{m,n}\ket m \frac{q_m}{q_n}\braket{m|n}\bra n=Id$
\\
\\
Note that a load of these arguments carry over into continuous spectra etc... aergtsertygtdhbrydjfyhbdyhcn
\subsection{Position and momentum basis - wavefunctions}
Have $[\hat x,\hat p]=i\hbar$ and say $\hat x\ket x=x\ket x$,  Cts spectrum in this case, so have
\begin{align*}
  \braket{x|x'}=\delta_{x-x'},
  \\
  Id=\int_\mathbb{R}dx\ket x \bra x
\end{align*}
similar for $\hat p$.  May represent a system in terms of either of these representations.  Often write $\psi(x)=\braket{x|\psi}$.

\end{document}
