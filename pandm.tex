\documentclass{article}
\usepackage{amsmath}
\usepackage{amssymb}
\usepackage{amsthm}

\usepackage{graphicx}

\newtheorem{theorem}{Theorem}[section]
\newtheorem{corollary}{Corollary}[theorem]
\newtheorem{lemma}[theorem]{Lemma}

\title{Probability and Measure}
\author{George Lee\\Girton College}
\begin{document}
\maketitle
\section{Measures}

\subsection{Definitions}
Let $E$ be a set.  A $\sigma$-$algebra$ $\mathcal{E}$ on $E$ is a collection of subsets of $E$ s.t.

\begin{align*}
  \forall n \in \mathbb{N}~A_n \in \mathcal{E} \implies\bigcup_\mathbb{N}A_n \in \mathcal{E},
  \\
  A\in\mathcal{E}\implies A^c \in \mathcal{E}
\end{align*}

Pair $(E,\mathcal{E})$ a measurable space, $\mathcal{E}$ the collection of measurable sets.
\\
$\mu:(E,\mathcal{E})\rightarrow[0,\infty]$ called a measure if for all disjoint $\{A_n\}_\mathbb{N}\subset\mathcal{E}$ we have that

$$
  \mu(\bigcup_\mathbb{N}A_n)=\sum_\mathbb{N}\mu(A_n).
$$

ie, countable additivity.  $(E,\mathcal{E},\mu)$ a measure space.

\subsection{Discrete measure theory}
Given $f:E\rightarrow[0,\infty]$ can do measure theory on measurable space $(E,2^E)$ via $\mu(A)=\sum_Af(a)$.

\subsection{Generated $\sigma$ - algebras}
Let $\mathcal{A}\subset2^E$.  Define
$$
  \sigma(\mathcal{A})=\bigcap\{\sigma\-~ algebras\supseteq\mathcal{A}\}
$$
Then (easy to check) $\sigma(\mathcal{A})$ a $\sigma$-algebra.

\subsection{$\pi$-systems and $d$-systems}
Let $\emptyset\in\mathcal{A}\subseteq2^E$. Have $\mathcal{A}$ a $\pi$-system if $A,B\in\mathcal{A}\implies A\cup B\in\mathcal{A}$.  We say that $E\in\mathcal{A}\subseteq2^E$ is a $d$-system if
\begin{align*}
  A,B\in\mathcal{A},A\subseteq B \implies B\backslash A \in \mathcal{A},
  \\
  (A_n)_\mathbb{N}\in\mathcal{A}^\mathbb{N},A_1\subseteq A_2\subseteq ... \implies \bigcup_\mathbb{N}A_n \in \mathcal{A}
\end{align*}
Note then that $\mathcal{A}$ both of these $\implies A$ a $\sigma$-algebra.
\begin{lemma}
  Dynkin's $\pi$ system lemma: Let $\mathcal{A}$ ba a $\pi$-system.  Then any $d$-system containing $\mathcal{A}$ also contains $\sigma(\mathcal{A})$.
\end{lemma}
\begin{proof}
  Let $\mathcal{D}=\bigcap\{d\text{-}systems\supseteq\mathcal{A}\}$.  Then $\mathcal{D}$ a $d$-system.  We show $\mathcal{D}$ also a $\pi$ system and thus a $\sigma$-algebra as required.  Consider
  $$
    \mathcal{D}'=\{B\in\mathcal{D} : \forall A \in \mathcal{A} : B\cap A\in\mathcal{D}\}\subseteq \mathcal{D}
  $$
  Then $\mathcal{D}\subseteq\mathcal{D}'$ ($\mathcal{A}$ a $d$-system).  Check $\mathcal{D}'$ a $d$-system: have $E\in\mathcal{D}'$, and let $B,C\in\mathcal{D}',~B\subseteq C$, then given $A\in\mathcal{A}$ we have
  \begin{align*}
    (C\backslash B)\cap A = (C\cap A)\backslash (B\cap A)\in\mathcal{D}
    \implies C\backslash B \in \mathcal{D}'
  \end{align*}
  Write $B_n \uparrow B$ if $B_1\subseteq B_2\subseteq...$ and $B=\bigcup_\mathbb{N}B_n$.  Let $(B_n)_\mathbb{N}\in\left.\mathcal{D}'\right.^\mathbb{N}$ be increasing, then $B_n\cap A \uparrow B\cap A$.  Thus $B\cap A \in \mathcal{D}\implies B\in\mathcal{D}'\implies B\in \mathcal{D}'\implies \mathcal{D}=\mathcal{D}'$.  Then let
  $$
    \mathcal{D}''=\{B\in\mathcal{D}:\forall A\in\mathcal{A}: B\cap A\in\mathcal{D}\}\subseteq \mathcal{D}
  $$
  $A,A'\in\mathcal{A}\implies A\cup A'\in \mathcal{D}$ and thus $\mathcal{A}\subseteq\mathcal{D}''$.  Check $\mathcal{D}''$ a $d$-system, like with $\mathcal{D}'$.  Then $\mathcal{D}''=\mathcal{D}\implies\mathcal{D}$ a $\pi$-system.
\end{proof}

\subsection{Set functions and properties}
Let $\emptyset\in\mathcal{A}\subseteq2^E$.  We call any $\mu:\mathcal{A}\rightarrow[0,\infty]$ with $\mu(\emptyset)=0$ a set function.  Let $\mu$ be such a function.
\begin{itemize}
  \item $(A,B\in\mathcal{A},A\subseteq B \implies \mu(A)\leq\mu(B))\implies \mu~increasing$.
  \item $(A,B,A\dot{\cup} B \in \mathcal{A}\implies \mu(A\cup B)=\mu(A)+\mu(B))\implies~\mu~additive$.
  \item $(A_1,A_2,...,\dot{\bigcup}_\mathbb{N} A_n \in \mathcal{A}\implies \mu(\bigcup_\mathbb{N}A_n)=\sum_\mathbb{N}\mu(A_n))\implies~\mu~countably~additive$.
\end{itemize}

\subsection{Construction of measures}
Let $\mathcal{A}\subseteq2^E$.  Say $\mathcal{A}$ a $ring~on~E$ if
\begin{itemize}
  \item $\emptyset\in\mathcal{A}$
  \item $A,B\in\mathcal{A}\implies B\backslash A,A\cup B\in\mathcal{A}$
\end{itemize}
Say $\mathcal{A}$ an $algebra~on~E$ if
\begin{itemize}
  \item $\emptyset\in\mathcal{A}$
  \item $A,B\in\mathcal{A}\implies A^c,A\cup B\in\mathcal{A}$
\end{itemize}
\begin{theorem}
  Carath\'eodory's extention theorem: Let $\mathcal{A}$ a ring on $E$ and $\mu$ a countably additive set function on $\mathcal{A}$.  Then can extend $\mu$ to a measure on $\sigma(\mathcal{A})$.
\end{theorem}
\begin{proof}
  Will write up later
\end{proof}
\subsection{Unhiqueness of measures}
\begin{theorem}
  Uniqueness of extention:srlgxdfkghdfg
\end{theorem}
\begin{proof}
  shtghfghcfgh
\end{proof}
\subsection{Borel sets and measures}
Let $(E,\tau)$ a topological space.  Then the Borel $\sigma$-algebra on $E$ is defined by $\mathcal{B}(E)=\sigma(\tau)$.  For $E=\mathbb R$ we may write $\mathcal B(E)=\mathcal{B}$.  A corresponding measure on such a space is called a Borel measure.  If further all compact sets have a finite measure we say it is a Radon measure.
\subsection{Probability measures, finite and $\sigma$-finite measures.}
For $(E,\mathcal E,\mu)$ a measure space, we say $\mu$ is a probability measure if $\mu(E)=1$, and tend to write instead $(\Omega,\mathcal{F},\mathbb{P})$.  $\mu(E)<\infty\implies\mu$ a finite measure, and finally if there is a countable collection of measurable sets of finite measure which cover $E$ then we say the space is $\sigma$-finite.
\subsection{Lebesgue measure}
\subsection{Existence of a non-Lebesgue-measurable subset of $\mathbb{R}$}
\subsection{Independence}
\subsection{Borel-Cantelli lemmas}
\section{Measurable functions and Random Variables}
\subsection{Measurable functions}
Let $(E,\mathcal E),(F,\mathcal F)$ be measurable spaces.  We say $f:E\rightarrow F$ is measurable if $U\in\mathcal F \implies f^{-1}U\in \mathcal E$.
\begin{itemize}
  \item $(G,\mathcal G)=(\mathbb R,\mathcal B)\implies f$ is a measurable function on $E$
  \item $(G,\mathcal G)=([0,\infty],\mathcal B([0,\infty]))\implies f$ is a non-negative measurable function on $E$
\end{itemize}
note neither definitions are more strict than the other.  Now seeing as inverse images preserve union and complementation, to show that a function is measurable it is sufficient to show that the inverse images of a collection of sets generating the $\sigma$-algebra are measurable.  For example in the case where $f$ is a measurable function on $E$ it suffices to show that the collection $\{f^{-1}(-\infty,x]:x\in\mathbb R\}=\{\{f(\omega)\leq x\}:x\in\mathbb R\}\subseteq\mathcal F$.  Note that if the measures are borel and the function is continuous it immediately follows it is measurable.
\\
\\
rtdhcyfujnyfgjhcvn
\\
\\
\begin{theorem}
  Monotone Class theorem: Let $(E,\mathcal E)$ be a measurable space and let $\mathcal A$ ba a $\pi$-system with $\sigma(\mathcal A)=\mathcal E$.  Let $\mathbb V$ be an $\mathbb{R}$-vector space of bounded $f:E\rightarrow\mathbb{R}$ such that the following hold:
  \begin{itemize}
    \item $x\mapsto1\in\mathbb V$
    \item $(f_n)_\mathbb{N}\in\mathbb V^\mathbb{N}$ with $0\leq f_n \uparrow f\implies f\in\mathbb V$
  \end{itemize}
  then ${bounded~measurable~functions}\subseteq\mathbb V$
\end{theorem}
\end{document}
